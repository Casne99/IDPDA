
\documentclass[a4paper,12pt]{report}


\usepackage[utf8]{inputenc}     % Per caratteri accentati
\usepackage[T1]{fontenc}        % Font encoding
\usepackage[italian]{babel}     % Lingua italiana
\usepackage{lmodern}            % Font leggibile
\usepackage{geometry}           % Margini personalizzati
\geometry{margin=3cm}
\usepackage{setspace}           % Spaziatura
\onehalfspacing                 % Interlinea 1.5
\usepackage{graphicx}           % Immagini
\usepackage{amsmath, amssymb}   % Matematica
\usepackage{hyperref}           % Link ipertestuali
\usepackage{fancyhdr}           % Intestazioni e piè di pagina


\usepackage{titlesec}


\titleformat{\chapter}[display]
  {\normalfont\huge\bfseries}  % Stile del titolo
  {}                           % Nessuna numerazione visibile
  {0pt}                        % Spazio tra numero e titolo (in questo caso, nessuno)
  {\Huge}                      % Stile applicato al titolo vero e proprio




% Impostazioni intestazioni
\pagestyle{fancy}
\fancyhead{}
\fancyhead[LE,RO]{\thepage}
\fancyhead[RE]{\leftmark}
\fancyhead[LO]{\rightmark}


% Frontespizio personalizzabile
\begin{document}


\begin{titlepage}
    \centering
    {\scshape\LARGE Università degli Studi di Milano \par}
    \vspace{1cm}
    {\scshape\Large Facoltà di \ldots\par}
    \vspace{0.5cm}
    {\scshape\large Corso di Laurea in Informatica\par}
    \vspace{2cm}
    {\huge\bfseries Input Driven Pushdown Automata ed Edit Distances\par}
    \vspace{2cm}
    \begin{flushleft}
        \textbf{Relatore:} Nome Relatore\\
        \textbf{Candidato:} Nome Cognome\\
        \textbf{Matricola:} 123456
    \end{flushleft}
    \vfill
    {\large Anno Accademico 2024--2025\par}
\end{titlepage}


\pagenumbering{Roman}
\tableofcontents
\clearpage


\pagenumbering{arabic}


\chapter{Introduzione}


Gli \emph{Input Driven Pushdown Automata}, anche detti \textit{IDPDA}, sono una classe di automi a pila in cui le operazioni sulla pila sono esclusivamente determinate dal simbolo di input. In particolare, il simbolo di input determina se l'automa compierà un'operazione di \emph{push} sulla pila, di \emph{pop}, oppure se lascerà la pila invariata.


Gli IDPDA risultano particolarmente utili in diversi ambiti, tra cui il \emph{parsing} dei linguaggi di programmazione, dei documenti XML e, più in generale, in tutti i contesti in cui la struttura dell'input riflette la struttura gerarchica dei dati.


È interessante notare come la famiglia dei linguaggi riconosciuti dagli IDPDA mantenga molte delle proprietà tipiche dei linguaggi regolari.


Un problema di particolare rilevanza, dove gli \textit{IDPDA} trovano utilizzo, consiste nel determinare la distanza tra una data stringa $s$ e un linguaggio $L$, detta anche \textit{distanza di edit}. Tale distanza è definita come il costo minimo di una sequenza di operazioni elementari di modifica su $s$ (sostituzione, inserimento, eliminazione di un simbolo), che la trasformano in una stringa $s' \in L$.


In uno scenario applicativo, si può supporre di aver ricevuto in input una stringa appartenente a $L$, la quale potrebbe tuttavia contenere errori. L'obiettivo è quindi ricostruire, con la miglior accuratezza possibile, la stringa originaria.


Per ottenere tale risultato, si può adottare un approccio a \emph{distanza minima}, ovvero determinare la stringa $y \in L$ tale che la \textit{distanza di edit} tra $x$ (la stringa ricevuta) e $y$ sia minima.


\chapter{Definzioni}
Una transizione di un IDPDA è determinata dal tipo di simbolo in input, appartenente all'alfabeto $\Sigma$, che stabilisce l'operazione da eseguire, eventualmente nulla.
Si può rappresentare l'alfabeto di input di un IDPDA come unione di tre sottoinsiemi disgiunti e finiti:


\begin{itemize}
  \item $\Sigma_{+1}$: parentesi sinistre, determinano una operazione di push sulla pila
  \item $\Sigma_{-1}$: parentesi destre, determinano una operazione di pop sulla pila
  \item $\Sigma_{0}$: simboli neutrali, non si traducono in operazioni sulla pila
\end{itemize}


Una stringa $s \in \Sigma$ è detta \textit{ben matchata} se ogni parentesi sinistra ha una parentesi destra corrispondente, e viceversa.

\section{Automi Input Driven Deterministici}



\chapter{Sviluppo}
Descrizione del lavoro svolto...


\chapter{Risultati}
Analisi e valutazione dei risultati...


\chapter{Conclusioni}
Osservazioni finali e possibili sviluppi futuri...


\appendix
\chapter{Appendice}
Eventuali tabelle, codice o materiale integrativo...


\clearpage
\begin{thebibliography}{9}


\bibitem{lamport}
  Leslie Lamport,
  \textit{LaTeX: A Document Preparation System},
  Addison Wesley, 1994.


% Aggiungere altre fonti


\end{thebibliography}


\end{document}

